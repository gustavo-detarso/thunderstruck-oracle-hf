% Created 2025-08-07 qui 16:34
% Intended LaTeX compiler: xelatex
\documentclass[11pt]{article}
\usepackage{graphicx}
\usepackage{longtable}
\usepackage{wrapfig}
\usepackage{rotating}
\usepackage[normalem]{ulem}
\usepackage{amsmath}
\usepackage{amssymb}
\usepackage{capt-of}
\usepackage{hyperref}
\usepackage[brazil, ]{babel}
\usepackage[utf8]{inputenc}
\usepackage[T1]{fontenc}
\usepackage[left=3cm, right=2cm, top=3cm, bottom=2cm]{geometry}
\sloppy
\hyphenpenalty=50
\tolerance=2000
\usepackage{graphicx}
\usepackage{sectsty}
\usepackage{ulem}
\renewcommand{\ULdepth}{1.8pt}
\author{Gustavo M. Mendes de Tarso}
\date{\today}
\title{}
\hypersetup{
 pdfauthor={Gustavo M. Mendes de Tarso},
 pdftitle={},
 pdfkeywords={},
 pdfsubject={},
 pdfcreator={Emacs 28.2 (Org mode 9.5.5)}, 
 pdflang={Pt_Br}}
\begin{document}

\begin{center}
\includegraphics[width=0.6\textwidth]{/home/gustavodetarso/Documentos/.share/png/logolula.png}
\end{center}

\vspace{-1.8cm}
\begin{center}
\textbf{Ministério da Previdência Social}\\
Secretaria do Regime Geral da Previdência Social\\
Departamento de Perícia Médica Federal
\end{center}

\vspace{-0.3cm}
\hrule

\vspace{-0.3cm}
\begin{center}
\textit{Coordenação-Geral de Assuntos Corporativos e Disseminação de Conhecimento}
\end{center}

\vspace{-0.8cm}
\begin{center}
\textbf{Gustavo Magalhães Mendes de Tarso}
\end{center}

\vspace{1.5cm}

\textbf{RELATÓRIO INSTITUCIONAL – SISTEMA thunderstruck-oracle-hf}

\section{Introdução}
\label{sec:orgf704b06}

O projeto thunderstruck-oracle-hf, também conhecido como Oráculo MPS, surge como uma resposta inovadora às crescentes demandas por eficiência e precisão na gestão de informações em sistemas públicos. Em um cenário onde a quantidade de dados disponíveis é vasta e em constante crescimento, a necessidade de ferramentas que possam processar, interpretar e utilizar essas informações de maneira eficaz é imperativa. O Oráculo MPS foi desenvolvido para atender a essa necessidade, utilizando tecnologias de ponta em processamento de linguagem natural e aprendizado de máquina para oferecer soluções automatizadas e inteligentes.

\section{Objetivos}
\label{sec:orga36bdad}

O principal objetivo do projeto é criar um sistema robusto que permita a geração automática de pares de perguntas e respostas (QA), o fine-tuning eficiente de modelos de linguagem de larga escala (LLM) e a busca semântica aprimorada. Além disso, o projeto visa implementar um sistema de geração aumentada por recuperação (RAG) que facilite a interação do usuário com o modelo, permitindo uma avaliação contínua e melhorias incrementais. Através dessas funcionalidades, o Oráculo MPS busca otimizar o acesso e a utilização de informações contidas em documentos oficiais, como editais e portarias, promovendo uma gestão pública mais eficiente e transparente.

\section{Dissertação}
\label{sec:orgdbe897b}

O projeto é estruturado em um fluxo de trabalho que abrange desde a geração de dados até o deploy do modelo treinado. Inicialmente, o sistema utiliza scripts como \texttt{gen\_qa\_chatgpt\_from\_txt.py} para converter documentos em formato de texto, como editais e portarias, em arquivos de QA automáticos no formato \texttt{.jsonl}. Este processo é fundamental para alimentar o sistema com dados estruturados que podem ser utilizados para treinamento e busca. Em seguida, o projeto emprega a indexação semântica através do uso de \texttt{faiss\_index.py}, que cria um índice FAISS para facilitar a busca rápida de contexto relevante. O treinamento do modelo é realizado com o script \texttt{autogen-model-hf.py}, que utiliza a técnica LoRA para um fine-tuning eficiente, permitindo que o modelo aprenda com um custo reduzido de recursos computacionais. Após o treinamento, o modelo é mesclado e preparado para deploy, estando disponível para inferência stand-alone. A busca RAG, implementada em \texttt{rag\_qa\_finetuned.py}, combina a recuperação de contexto com a geração de respostas, oferecendo uma interface de linha de comando para interação com o usuário.

\section{Expectativas}
\label{sec:org9170a2f}

Espera-se que o Oráculo MPS traga significativos avanços na forma como informações são geridas e acessadas no setor público. A automatização da geração de QA e a busca semântica eficiente têm o potencial de reduzir o tempo e os recursos necessários para a interpretação de documentos complexos, como editais de concursos e portarias governamentais. Além disso, a capacidade de fine-tuning eficiente permite que o sistema se adapte rapidamente a novos dados e necessidades, garantindo que as informações estejam sempre atualizadas e relevantes. A implementação de um sistema RAG com feedback do usuário também é vista como uma oportunidade para melhorar continuamente a precisão e a utilidade das respostas geradas pelo sistema.

\section{Conclusão}
\label{sec:orgbd4b4ac}

O projeto thunderstruck-oracle-hf representa um passo significativo em direção à modernização e otimização dos sistemas de gestão de informações no setor público. Ao integrar tecnologias avançadas de processamento de linguagem natural e aprendizado de máquina, o Oráculo MPS oferece uma solução abrangente e eficiente para os desafios enfrentados na interpretação e utilização de grandes volumes de dados. Com expectativas de impacto positivo na eficiência administrativa e na transparência governamental, o projeto se posiciona como uma ferramenta essencial para o futuro da gestão pública.
\end{document}